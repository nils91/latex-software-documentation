%%%%%%%%%%%%%%%%%%%%%%%%%%%%%%%%%%%%%%%%%%%%%%%%%%%%%%%%%%%%%%%%%%%%%%%
%% Vorlage fuer Abschlussarbeit (IFM, FB-T, FH Bielefeld)
%% Autor: Nils Dralle
%% Datum: 10.06.2020
\documentclass[11pt, a4paper]{article}




%%%%%%%%%%%%%%%%%%%%%%%%%%%%%%%%%%%%%%%%%%%%%%%%%%%%%%%%%%%%%%%%%%%%%%%
%% Nuetzliche Pakete
\usepackage[ngerman]{babel}
\usepackage[utf8]{inputenc}
\usepackage[T1]{fontenc}
\usepackage{lmodern}   
% \usepackage{times}

\usepackage[babel]{csquotes}


%------------------------------------- Seitenraender ----------------
\usepackage[left=2cm,right=2cm,top=2cm,bottom=2.5cm]{geometry}
%------------------------------------- Seitenraender ----------------





% \usepackage{latexsym}
% \usepackage{exscale}
% \usepackage{verbatim}
% 
% \usepackage{tikz}
% \usepackage{wasysym}
\usepackage{graphicx}
\usepackage{subfigure}
\usepackage{listings}
% \usepackage{fancybox}
\usepackage{pdfpages}
% \usepackage{xspace}
\usepackage{hyperref}
\usepackage{color}
% \usepackage[table]{xcolor}
% \usepackage{colortbl}
% \usepackage{supertabular}
\usepackage{longtable}
% \usepackage{colortbl}
% \usepackage{rotating}
\usepackage[ruled,vlined]{algorithm2e}
\usepackage{float}
% 

\usepackage{amssymb,amsmath}

\definecolor{listinggray}{gray}{0.92}
\definecolor{commentgray}{gray}{0.6}
\definecolor{quellengray}{gray}{0.8}


\lstnewenvironment{java}{
%\lstset{xleftmargin=\bigskipamount, xrightmargin=\bigskipamount, aboveskip=\bigskipamount, belowskip=\bigskipamount,}
\lstset{language=Java, basicstyle=\footnotesize\ttfamily\mdseries}
\lstset{keywordstyle=\bfseries\color{blue},identifierstyle=\ttfamily, commentstyle=\bfseries\color{gray}\textsl,stringstyle=\color{magenta}\upshape}
\lstset{emphstyle=\color{red}, emphstyle={[2]\color{blue}}}
%\lstset{texcl=true}
\lstset{texcl=false}
\lstset{escapechar=@}
\lstset{inputencoding=utf8}
%\lstset{framexleftmargin=5mm, frame=shadowbox, rulesepcolor=\color{black}}
\lstset{columns=fullflexible, showspaces=false, showstringspaces=false, numbers=none, breaklines=true,tabsize=4, backgroundcolor=\color{listinggray}}
}{}




% \usepackage{makeidx}         % index
% \makeindex
% % damits angezeigt wird:
% % makeindex diss.idx
% \usepackage[german]{nomencl} % nomenclatur
% \makeglossary
% % damits angezeigt wird:
% % makeindex diss.glo -s nomencl.ist -o diss.gls
% 
% \usepackage{theorem}
% \usepackage{varioref}  % \vref{} ...


% % \sloppy




%%%%%%%%%%%%%%%%%%%%%%%%%%%%%%%%%%%%%%%%%%%%%%%%%%%%%%%%%%%%%%%%%%%%%%%
%% Start des Dokuments
\begin{document}








%%%%%%%%%%%%%%%%%%%%%%%%%%%%%%%%%%%%%%%%%%%%%%%%%%%%%%%%%%%%%%%%%%%%%%%
%% Titelseite
%%%%%%%%%%%%%%%%%%%%%%%%%%%%%%%%%%%%%%%%%%%%%%%%%%%%%%%%%%%%%%%%%%%%%%%
%% Titelseite
% \maketitle    % Anzeige der Standardtitelseite
\begin{titlepage}
\thispagestyle{empty}
    \hrule
    \vspace{1cm}
    \begin{center}
        {\huge\bf\sc Documentaion for ...}
    \end{center}
    \vfill
    \begin{Large}
        {\bf Documentation},\\[48pt]
        <Product>\\
        <Company>\\[72pt]
        written by <Name>,\\
        Version <Version> on Date <Date>
        \vfill
        \noindent{\today}
    \end{Large}
    \vspace{1cm}
    \hrule
\end{titlepage}


\clearpage


%%%%%%%%%%%%%%%%%%%%%%%%%%%%%%%%%%%%%%%%%%%%%%%%%%%%%%%%%%%%%%%%%%%%%%%
%% Abstract
%%%%%%%%%%%%%%%%%%%%%%%%%%%%%%%%%%%%%%%%%%%%%%%%%%%%%%%%%%%%%%%%%%%%%%%
%% Abstract
\begin{abstract}
\thispagestyle{empty}

  
\end{abstract}


\clearpage


%%%%%%%%%%%%%%%%%%%%%%%%%%%%%%%%%%%%%%%%%%%%%%%%%%%%%%%%%%%%%%%%%%%%%%%
%% Verzeichnisse
%%%%%%%%%%%%%%%%%%%%%%%%%%%%%%%%%%%%%%%%%%%%%%%%%%%%%%%%%%%%%%%%%%%%%%%
%% Verzeichnisse

\pagenumbering{roman}
\setcounter{page}{1}

\tableofcontents
\clearpage

\listoffigures
\clearpage

\listoftables
\clearpage

% \listofalgorithms
% \clearpage



% Absätze nicht einrücken, dafür aber mehr Platz dazwischen ...
% sollte NACH den Inhaltsverzeichnissen kommen!!!
\setlength{\parindent}{0pt}
\setlength{\parskip}{1.5ex plus 0.5ex minus 0.2ex}
\newpage
\pagenumbering{arabic}
\setcounter{page}{1}


\clearpage


%%%%%%%%%%%%%%%%%%%%%%%%%%%%%%%%%%%%%%%%%%%%%%%%%%%%%%%%%%%%%%%%%%%%%%%
%% Einleitung
%%%%%%%%%%%%%%%%%%%%%%%%%%%%%%%%%%%%%%%%%%%%%%%%%%%%%%%%%%%%%%%%%%%%%%%
%% Einleitung
\section{Einleitung}
\label{sec:einleitung}
\clearpage

%%%%%%%%%%%%%%%%%%%%%%%%%%%%%%%%%%%%%%%%%%%%%%%%%%%%%%%%%%%%%%%%%%%%%%%
%% Chapters
\section{Konzept}
\label{sec:concept}\clearpage

\section{Implementation}
\label{sec:implementation}\clearpage


%%%%%%%%%%%%%%%%%%%%%%%%%%%%%%%%%%%%%%%%%%%%%%%%%%%%%%%%%%%%%%%%%%%%%%%
%% Zusammenfassung und Ausblick
%%%%%%%%%%%%%%%%%%%%%%%%%%%%%%%%%%%%%%%%%%%%%%%%%%%%%%%%%%%%%%%%%%%%%%%
%% Zusammenfassung und Ausblick
\section{Ausblick}
\label{sec:ausblick}\clearpage

%%%%%%%%%%%%%%%%%%%%%%%%%%%%%%%%%%%%%%%%%%%%%%%%%%%%%%%%%%%%%%%%%%%%%%%
%% Literatur
%% braucht einen Zwischenlauf mit "bibtex abschlussarbeit"
%% nach Änderungen: pdflatex, bibtex, pdflatex, pdflatex, pdflatex
%%%%%%%%%%%%%%%%%%%%%%%%%%%%%%%%%%%%%%%%%%%%%%%%%%%%%%%%%%%%%%%%%%%%%%%
%% Literatur
\section{Literaturverzeichnis}
\label{sec:literatur}

    \bibliography{babib}    % braucht einen Zwischenlauf mit "bibtex abschlussarbeit"
    \bibliographystyle{alpha}   % normales bibtex; [abk.nam+jahr]
%     \bibliographystyle{alphadin}   % bibtex nach DIN [abk.nam+jahr]
%     \bibliographystyle{babalpha}  % bibtex mit babelbib
%     \bibliographystyle{chicago} % (name, jahr)
%     \bibliographystyle{splncs}  % nur (1)
%     \bibliographystyle{named}
    
    
%     TODO anpassen
    % diese hier zusätzlich auf alle Faelle ins Literaturverzeichnis
    %\nocite{Klatzky98a,Levelt86,Levinson1996,Mukerjee1998}
\nocite{Ullenboom2018} %Java-Buch

    
    
    \clearpage


%%%%%%%%%%%%%%%%%%%%%%%%%%%%%%%%%%%%%%%%%%%%%%%%%%%%%%%%%%%%%%%%%%%%%%%
%% Anhang
\begin{appendix}

%%%%%%%%%%%%%%%%%%%%%%%%%%%%%%%%%%%%%%%%%%%%%%%%%%%%%%%%%%%%%%%%%%%%%%%
%% Appendix
\begin{appendix}


   
    \section{Erster Anhang}
    
    \section{Noch ein Anhang}
    
    
\end{appendix}


\clearpage

\end{appendix}


%%%%%%%%%%%%%%%%%%%%%%%%%%%%%%%%%%%%%%%%%%%%%%%%%%%%%%%%%%%%%%%%%%%%%%%
%% Game over ;-)
\end{document}


